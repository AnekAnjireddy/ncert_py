\documentclass{article}
\usepackage{amsmath,amssymb,amsfonts,amsthm}

\begin{document}

\providecommand{\pr}[1]{\ensuremath{\Pr\left(#1\right)}}
\providecommand{\brak}[1]{\ensuremath{\left(#1\right)}}
\newcommand{\solution}{\noindent \textbf{solution: }}

\textbf{Question 11.16.4.9}\\
If 4-digit numbers greater than 5,000 are randomly formed from the digits 0, 1, 3, 5, and 7, what is the probability of forming a number divisible by 5 when:
\begin{enumerate}
    \item The digits are repeated?
    \item The repetition of digits is not allowed?
\end{enumerate}

\solution

Let's solve each part separately:

\textbf{(i) Digits are Repeated}\\
Let number of favourable outcomes be N(A) and total outcomes be N(T). \\
For the number to be greater than  5000, the 1000s digit must be 5 or 7 and there is no constraint on the other digits. And we must exclude the case '5000'.\\
Hence,
\begin{align}
	N(T)= (2 \times 5 \times 5 \times 5)-1=249
\end{align}

For the number to be divisible by 5, the last digit must be either 0 or 5.  Here also we must exclude the case '5000'\\
Hence,

\begin{align}
	N(A)= (2 \times 5 \times 5 \times 2)-1=99
\end{align}

The probability of forming a number divisible by 5 when digits are repeated is:
\begin{align}
P(\text{divisible by 5}) = \frac{\text{N(A)}}{\text{N(T)}} = \frac{99}{249} = \frac{33}{83}.
\end{align}


\textbf{(ii) No Repetition of Digits}\\
Let number of favourable outcomes be N(B) and total outcomes be N(T). \\
Hence,
\begin{align}
	N(T)= 2 \times 4 \times 3 \times 2=48
\end{align}

For the number to be divisible by 5, the last digit must be either 0 or 5. \\
The favourable cases are:
\begin{enumerate}
\item If the 4-digit number starts with 5, the last digit must be 0.
	\begin{align}
		N_{1}= 1 \times 3 \times 2 \times 1=6
	\end{align}
\item if the 4-digit number starts with 7, the last digit can be 0 or 5
	\begin{align}
		N_{2}= 1 \times 3 \times 2 \times 2=12
	\end{align}
	
\end{enumerate}

The total number of favourable cases N(A) will be:
	\begin{align}
		N(B)=N_{1}+N_{2}= 18
	\end{align}
	
The probability of forming a number divisible by 5 when digits are not repeated is:

\begin{align}
P(\text{divisible by 5}) = \frac{\text{N(A)}}{\text{N(T)}} = \frac{18}{48} = \frac{3}{8}.
\end{align}

\end{document}



