\documentclass{article}
\usepackage{amsmath,amssymb,amsfonts,amsthm}

\begin{document}

\providecommand{\pr}[1]{\ensuremath{\Pr\left(#1\right)}}
\providecommand{\brak}[1]{\ensuremath{\left(#1\right)}}
\newcommand{\solution}{\noindent \textbf{solution: }}

\textbf{Question 11.16.4.9}\\
If 4-digit numbers greater than 5,000 are randomly formed from the digits 0, 1, 3, 5, and 7, what is the probability of forming a number divisible by 5 when:
\begin{enumerate}
    \item The digits are repeated?
    \item The repetition of digits is not allowed?
\end{enumerate}

\solution
Let $X$ be a random variable such that 
\begin{align}
	X = \begin{cases}
		0 &  n<5000\\
		1 &  n>5000\end{cases}
\end{align}

Let $Y$ be a random variable such that:
\begin{align}
	Y = \begin{cases}
		0 & n \not\equiv 0 \pmod{5}\\
		1 & n \equiv 0 \pmod{5}\end{cases}
\end{align}

We need to find \pr{Y \mid X} in each case. \\

\begin{align}
	\pr{Y \mid X}=\frac{\pr{XY} }{\pr{X}}
\end{align}

Let's solve each part separately. \\

\textbf{(i) Digits are Repeated}\\
For n to be greater than 5000, first digit must be 5 or 7. Hence,
\begin{align}
	\pr{X=1} &=\frac{249}{625}
\end{align}

For n to be greater than 5000 and also divisble by 5:
\begin{enumerate}
\item first digit must be 5 or 7.
\item last digit must be 5 or 0.
\end{enumerate}
Hence,

\begin{align}
	\pr{XY=1} &=\frac{99}{625}
\end{align}

With this information we can find the required answer,
\begin{align}
	\pr{Y \mid X=1}=\frac{\pr{XY=1} }{\pr{X=1}}\\
	\pr{Y \mid X=1}=\frac{99}{249}\\	
	\implies \pr{Y \mid X=1}=\frac{33}{83}	
\end{align}


\textbf{(ii) No Repetition of Digits}\\
For n to be greater than 5000, first digit must be 5 or 7. Hence,
\begin{align}
	\pr{X=1} &=\frac{48}{120}
\end{align}

The cases for n to be greater than 5000 and also divisble by 5:
\begin{enumerate}
\item first digit must be 5 and last digit must be 0.
\item first digit must be 7 and last digit must be 5 or 0.
\end{enumerate}
Hence,

\begin{align}
	\pr{XY=1} &=\frac{18}{120}
\end{align}

With this information we can find the required answer,
\begin{align}
	\pr{Y \mid X=1}=\frac{\pr{XY=1} }{\pr{X=1}}\\
	\pr{Y \mid X=1}=\frac{18}{48}\\	
	\implies \pr{Y \mid X=1}=\frac{3}{8}	
\end{align}

\end{document}



